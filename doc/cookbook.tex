\documentclass[a4paper,11pt]{article}
\usepackage[utf8]{inputenc}
\usepackage[T1]{fontenc}
\usepackage[francais]{babel}
\usepackage{lettrine}
\usepackage{tikz}
\usetikzlibrary{shapes}
\usepackage{oldgerm}
\usepackage{hyperref}

\usepackage{geometry}
\geometry{margin=1cm}

\usetikzlibrary{arrows}

\newtheorem{definition}{Définition}
\newtheorem{foo}{}

\hypersetup{
    bookmarks=true,         % show bookmarks bar?
    pdftoolbar=true,        % show Acrobat’s toolbar?
    pdfmenubar=true,        % show Acrobat’s menu?
    pdfnewwindow=true      % links in new PDF window
}

\title {MIP recommandations Cookbook}

\begin{document}

\maketitle

\section{Que faire quand on dispose du catalogue}

\begin{enumerate}
\item Importer les produits;
\item Calculer les similarités entre produits
\item Calculer les similarités entre produits et segments
\item Calculer la liste ordonnée des produits par segment
\end{enumerate}

\subsection{Importer les produits}
A la reception du catalogue, on l'importe dans la base mip.
Si le catalogue est dans le fichier \emph{products.2016.csv}:

\begin{verbatim}
drop table if exists catalog_2016;
create table catalog_2016 (like catalog_pattern);
\copy catalog_2016 from '/home/thierry/MIP/sql/products.2016.csv' 
with CSV delimiter ',' quote '"' HEADER;
select create_products('catalog_2016', 'products_2016');
\end{verbatim}

A ce point les produits sont dans la table products\_2016. On calcul le fichier global des produits par concaténation avec les produits des années précédentes:

\begin{verbatim}
drop table if exists products_total;
create table products_total as (select * from products_2013 union select * from products_2014 union select * from products_2015 union select * from products_2016);
\end{verbatim}

\subsection{Calculer les similarités entre produits}
\begin{verbatim}
select create_products_profiles ('products_2016', 'products_profiles_2016');
select create_products_similarities_from_profiles 
('products_profiles_2016', 'products_cosim_0_2016');
\end{verbatim}


\subsection{Calculer les similarités entre produits et segments}
\begin{verbatim}
select create_buyers_or_segments_products_similarities (
'segments_profiles', 
'products_profiles_2016', 
'segments_products_similarities_2016');
\end{verbatim}

\subsection{calculer les listes par segment}
\subsubsection{Calculer les listes pour le segment 0}
\begin{verbatim}
-- PT0
-- segment 0
-- create the rank value
drop table if exists seg0reco2;
create table seg0reco2 as
(select *, rank () over (partition by ka order by co_usage desc, co_sim desc, tiebreaker) as rnk
from (select *, random () as tiebreaker from products_cosim_2016 where ka != kb) A);

-- gather values  in vectors
drop table if exists seg0reco3;
create table seg0reco3 as
(select 0 as segid, ka, array_agg(kb order by rnk)::integer[] as kbs from seg0reco2 group by ka);
\end{verbatim}

\subsubsection{Calculer la liste ordonée des produits par segment}
\begin{verbatim}
-- join co-sim and similarities table
drop table if exists reco1;
create table reco1
as (select segid, ka, A.kb,  co_usage, co_sim, sim, random()
as tiebreaker from products_cosim_2016 A , segments_products_similarities_2016 B 
where A.kb = B.kb and A.ka != A.kb);

-- order recom by rankers
drop table if exists reco2;
create table reco2 as (
select *, rank() over (partition by segid, ka order by co_usage desc, co_sim desc, sim desc, 
tiebreaker) as rnk from reco1);

-- gathers recommendations in a vector
drop table if exists reco3;
create table reco3 as 
(select segid, ka, array_agg(kb order by rnk)::integer[] as kbs from reco2 group by segid, ka);

--on peut concatener les tables reco3 et seg0reco3 pour avoir les recommandations du segment 0 sous la main facilement:

create table reco4 as (select * from seg0reco3 union select * from reco3);
alter table reco4 add primary key (segid, ka);


\end{verbatim}

Voir \emph{co-usage-proxy.sql} pour calculer un cousage basé par le cousage d'une année précédente.

\subsubsection{reco1 format}
\begin{tabular}{ll}
segid      & integer          \\
 ka         & integer         \\ 
 kb         & integer         \\ 
 co\_usage   & real           \\ 
 co\_sim     & real           \\ 
 sim        & real            \\ 
 tiebreaker & double precision 
\end{tabular}

\subsubsection{reco2 format}
\begin{tabular}{ll}
segid      & integer          \\
 ka         & integer         \\ 
 kb         & integer         \\ 
 co\_usage   & real           \\ 
 co\_sim     & real           \\ 
 sim        & real            \\ 
 tiebreaker & double precision \\
 rnk        & bigint
\end{tabular}

\subsubsection{reco3 format}
\begin{tabular}{lll}
 segid  & integer   & not null \\
 ka     & integer   & not null \\
 kbs    & integer[] & 
\end{tabular}

Indexes:
    "reco3\_pkey" PRIMARY KEY, btree (segid, ka)


\section{Prise en compte des buyers}

\begin{enumerate}
\item Importer le fichier;
\item constituer le fichier total des buyers par concaténation avec l'historique
\item calculer les profils des buyers;
\item calculer la table des segments
\end{enumerate}

Le calcul du profil des buyers prend en compte l'integralité du fichier reelport, pas seulement les produits
pour lesquels il y a eut un screening.

\begin{verbatim}
-- Import
drop table if exists reelport_2016;
create table reelport_2016 (like reelport_pattern);
\copy reelport_2016 from '/home/thierry/MIP/data/extract-2016.csv' with CSV delimiter E'\t' quote '"' HEADER;
drop table if exists reelport_all;

create table reelport_all as (select A.* from (select * from reelport_2013 union select * from reelport_2014 union select * from reelport_2015) A, (select distinct buyerid from reelport_2016) B where A.buyerid = B.buyerid);

select create_activities ('reelport_2016', 'activities_2016');

select create_buyers_profiles ('reelport_2016', 'products_total', 'buyers_profiles_2016');
alter table buyers_profiles_2016 add primary key (buyerid);

\end{verbatim}

Le segmentation et le calcul de la table buyers => segments et fait en R:
\begin{verbatim}
  createClusterTable <- function (con
  , learnTable         # profils pour apprendre les clusters
  , predictTable       # profils à segmenter
  , newSegmentTable,   # table de mapping buyer => segment
  k=20)
\end{verbatim}

\begin{verbatim}
select create_cousage ('activities_2016', 'products_cousage_2016');
select merge_products_cosim_cousage ('products_cosim_0_2016', 'products_cousage_2016', 
'products_cosim_2016');
-- ici on reprend le calcul PT0
-- ...
-- jointure avec les buyers:
drop index if exists reco5_idx;
drop table if exists reco5;
create table reco5 as (select buyerid, A.ka, kbs from reco4 A, buyers_segments B 
where A.segid = B.segid);
create index reco5_idx on reco5 (buyerid);
-- on supprime des recommendations les items préalablement touchés
drop table if exists toucheditems;
create table toucheditems as (select substring(buyerid, 3)::integer as buyerid, 
array_agg (screeningid)::integer[] as touched from activities_2016 group by buyerid);
alter table toucheditems add primary key (buyerid);
--- 
create table reco6 as (select A.buyerid, 
case when touched is null then kbs else my_array_diff (kbs, touched) end as kbs 
from reco5 A left outer join  toucheditems B on (A.buyerid = B.buyerid));

\end{verbatim}

TODO : quid des buyers non segmentés ? s'assurer que tout le monde à des reco

\section{Tables}
 \subsection{Table des profils des produits}

\begin{tabular}{lll}
 screeningnumber    & smallint & not null \\
 agegroup\_preschool & integer & \\
 agegroup\_toddler   & integer & \\
 agegroup\_kids      & integer & \\
 agegroup\_tnt       & integer & \\
 agegroup\_family    & integer & \\
 genre\_animation    & integer & \\
 genre\_liveaction   & integer & \\
 genre\_education    & integer & \\
 genre\_featurefilm  & integer & \\
 genre\_art          & integer & \\
 genre\_game         & integer & \\
 genre\_shorts       & integer & \\
 genre\_other        & integer & \\
 boys              & integer  & \\
 girls             & integer  & 
\end{tabular}

Indexes:
    "products\_profiles\_2015\_pkey" PRIMARY KEY, btree (screeningnumber)

\subsection{Table des products}
\begin{tabular}{lll}
productid                              & integer           & \\ 
 companyparticipantid                   & integer           & \\ 
 title                                  & character varying & \\ 
 description                            & character varying & \\ 
 programmesprojectsips                  & character varying & \\ 
 screeningnumber                        & smallint          & not null \\
 new                                    & boolean           & \\ 
 companyatoz                            & character(1)      & \\ 
 genremipjunior                         & character varying & \\ 
 yearofproduction                       & character varying & \\ 
 programmesproductionstatus             & character varying & \\ 
 format                                 & character varying & \\ 
 agegroup                               & character varying & \\ 
 numberofepisodes                       & character varying & \\ 
 salesrightscountries                   & character varying & \\ 
 numberofseriesseasons                  & character varying & \\ 
 licencingtarget                        & character varying & \\ 
 programmecreditsauthors                & character varying & \\ 
 programmecreditsdirector               & character varying & \\ 
 programmecreditsproducer               & character varying & \\ 
 programmecreditsmainbroadcaster        & character varying & \\ 
 projectcreditscoproductionpartners     & character varying & \\ 
 projectcreditsadditionalpartners       & character varying & \\ 
 projectcreditsproductionbudget         & character varying & \\ 
 projectcreditsfinancestillrequired     & character varying & \\ 
 projectcreditsproductioncompletiondate & character varying & \\ 
 salesrightsplatform                    & character varying & \\ 
 languagesavailable                     & character varying & \\ 
 yearofproductionstartdate              & smallint          & \\ 
 yearofproductionenddate                & smallint          & \\ 
 format1lengthinminutes                 & smallint          & \\ 
 format2lengthinminutes                 & smallint          & \\ 
 format1numberofepisodes                & smallint          & \\ 
 format2numberofepisodes                & smallint          & \\ 
 numberofcopies                         & smallint          & \\ 
 numberofviewsdownloaded                & smallint          & \\ 
 licensingaudiencetype                  & character varying & \\ 
 dummy                                  & character(1)      & \\ 
 languages\_arr                          & text[]            & \\ 
 formats\_arr                            & text[]            & \\ 
 platforms\_arr                          & text[]            & \\ 
 freshyearprod                          & smallint          &
\end{tabular}

Indexes:
    "products\_2015\_pkey" PRIMARY KEY, btree (screeningnumber)

\subsection{Table des profils de segments}
\begin{tabular}{ll}
segid                & integer          \\ 
 weight               & double precision \\ 
 agegroup\_preschool   & numeric          \\ 
 agegroup\_toddler     & numeric          \\ 
 agegroup\_kids        & numeric          \\ 
 agegroup\_tnt         & numeric          \\ 
 agegroup\_family      & numeric          \\ 
 genre\_animation      & numeric          \\ 
 genre\_liveaction     & numeric          \\ 
 genre\_education      & numeric          \\ 
 genre\_featurefilm    & numeric          \\ 
 genre\_art            & numeric          \\ 
 genre\_game           & numeric          \\ 
 genre\_shorts         & numeric          \\ 
 genre\_other          & numeric          \\ 
 boys                 & numeric          \\ 
 girls                & numeric          \\ 
 p\_agegroup\_preschool & double precision \\ 
 p\_agegroup\_toddler   & double precision \\ 
 p\_agegroup\_kids      & double precision \\ 
 p\_agegroup\_tnt       & double precision \\ 
 p\_agegroup\_family    & double precision \\ 
 p\_genre\_animation    & double precision \\ 
 p\_genre\_liveaction   & double precision \\ 
 p\_genre\_education    & double precision \\ 
 p\_genre\_featurefilm  & double precision \\ 
 p\_genre\_art          & double precision \\ 
 p\_genre\_game         & double precision \\ 
 p\_genre\_shorts       & double precision \\ 
 p\_genre\_other        & double precision \\ 
 p\_boys               & double precision \\ 
 p\_girls              & double precision 
\end{tabular}

\subsection{Table des profils de segments}

\begin{tabular}{lll}
buyerid              & integer  & not null \\
 weight               & real    & \\ 
 agegroup\_preschool   & smallint & \\ 
 agegroup\_toddler     & smallint & \\ 
 agegroup\_kids        & smallint & \\ 
 agegroup\_tnt         & smallint & \\ 
 agegroup\_family      & smallint & \\ 
 genre\_animation      & smallint & \\ 
 genre\_liveaction     & smallint & \\ 
 genre\_education      & smallint & \\ 
 genre\_featurefilm    & smallint & \\ 
 genre\_art            & smallint & \\ 
 genre\_game           & smallint & \\ 
 genre\_shorts         & smallint & \\ 
 genre\_other          & smallint & \\ 
 boys                 & smallint & \\ 
 girls                & smallint & \\ 
 p\_agegroup\_preschool & real     & \\ 
 p\_agegroup\_toddler   & real     & \\ 
 p\_agegroup\_kids      & real     & \\ 
 p\_agegroup\_tnt       & real     & \\ 
 p\_agegroup\_family    & real     & \\ 
 p\_genre\_animation    & real     & \\ 
 p\_genre\_liveaction   & real     & \\ 
 p\_genre\_education    & real     & \\ 
 p\_genre\_featurefilm  & real     & \\ 
 p\_genre\_art          & real     & \\ 
 p\_genre\_game         & real     & \\ 
 p\_genre\_shorts       & real     & \\ 
 p\_genre\_other        & real     & \\ 
 p\_boys               & real     & \\ 
 p\_girls              & real     & 
\end{tabular}

Indexes:
    "buyers\_profiles\_2013\_2015\_pkey" PRIMARY KEY, btree (buyerid)

\subsection{Table des similarités entre produits}

\begin{tabular}{lll}
 ka                    & integer & not null \\
 kb                    & integer & not null \\
 co\_agegroup\_preschool & integer & \\
 co\_agegroup\_toddler   & integer & \\
 co\_agegroup\_kids      & integer & \\
 co\_agegroup\_tnt       & integer & \\
 co\_agegroup\_family    & integer & \\
 co\_genre\_animation    & integer & \\
 co\_genre\_liveaction   & integer & \\
 co\_genre\_education    & integer & \\
 co\_genre\_featurefilm  & integer & \\
 co\_genre\_art          & integer & \\
 co\_genre\_game         & integer & \\
 co\_genre\_shorts       & integer & \\
 co\_genre\_other        & integer & \\
 co\_boys               & integer & \\
 co\_girls              & integer & \\
 co\_sim                & real    & \\
 co\_play               & integer & \\
 co\_screen             & integer & \\
 co\_usage              & real    &
\end{tabular}

Indexes:
    "products\_cosim\_2015\_pkey" PRIMARY KEY, btree (ka, kb)

\section{Remarques}

On peut retrouver les buyersid dans le fichier 'participating' ainsi:

\begin{verbatim}
update participants_2015 set norm_personid = regexp_replace (x103381_eticketurl_en, '.+Contact=(\d+)', E'\\1')::integer;
\end{verbatim}

la selection des buyers se fait en filtrant sur le champs: x102791\_role.

De la on peut connaître le pays du buyer, si necessaire.

\end{document}
